%%
%% Front page for ECE MS Thesis / PhD Dissertation Description 
%% based on `pdf-toc.tex' from pdfpages 
%%   http://www.ctan.org/pkg/pdfpages

% note added special options for TOC so taht the partial toc works
\documentclass{article}
%\documentclass[toc=flat,numbers=noenddot]{scrreprt}
\usepackage{pdfpages}
\usepackage{verbatim}
\usepackage[plainpages=false, pdfpagelabels,
            bookmarksopen]{hyperref}
						
% overwrite footer definition
\usepackage{fancyhdr}		
\usepackage{lastpage}		
\usepackage{pdflscape}


\title{Electrical and Computer Engineering\\MS Thesis / PhD Dissertation Template\\Instructions}

\hypersetup{				% setup PDF information fields
pdfauthor = {Gunar Schirner},
pdftitle = {ECE MS Thesis / PhD Dissertation Template Instructions},
pdfsubject = {},
pdfkeywords = {}
}


\pagestyle{fancy}

\fancyhf{} % clear all header and footers
\fancypagestyle{plain}{}
% use same footer for chapter pages and normal pages
\renewcommand{\headrulewidth}{0pt} % remove the header rule

\begin{document}
\maketitle

\tableofcontents


\section{Instructions}


These are the MS thesis and PhD dissertation LaTeX templates for
the ECE Department, Northeastern University.

\begin{enumerate}
\item The \textbf{same} template is used for MS thesis and PhD dissertation. By default, a MS thesis is produced. To change to a PhD Dissertation modify the file thesis.tex and change the line:
\vspace{-8pt}
\begin{verbatim}
\documentclass[]{macro/neu_msthesis}
\end{verbatim}
\vspace{-8pt}
to
\vspace{-8pt}
\begin{verbatim}
\documentclass[PHD]{macro/neu_msthesis}
\end{verbatim}
\vspace{-8pt}

The examples for MS Thesis (see Section \ref{sec:ms_thesis}) and PhD Dissertation (see Section \ref{sec:phd_diss}) have been build using this template.

\item The file thesis.tex allows easy configuration of all common aspects of the thesis/dissertation. Examples include: author name, title, number and names of committee members. Please see the definitions and descriptions in thesis.tex.

\item Please put all style files that you need to use in the macro
folder and call them using the macro.tex file in this folder.

\item All latex source files corresponding to different chapters go in
the tex folder. All figures go in the fig folder and all
bibliography-related files in the bib folder.

\item Both pdf and eps graphics can be used; although pdf is
preferable. If eps files are used, they will be converted to pdf on
the fly. It is recommended that you run your source file with
pdflatex (alternatively you can use latex+dvips+ps2pdf but you
should be aware that this path cannot handle pdf graphics files.)

\item You need to run bibtex and makeindex to generate the bibliography
and the index.

\item CTAN (the Comprehensive TeX Archive Network) is the source of
almost everything that you need for your thesis. You can access
CTAN at http://www.ctan.org. The following packages are needed,
if you do not already have them, make sure to download them from
CTAN and put them in a path that your LaTeX implementation can 
find them:

\begin{itemize}
	\item amsfonts
	\item amssymb
	\item amsmath
	\item times
	\item bm
	\item multirow
	\item makeidx
	\item acronym
	\item url
	\item graphicx
	\item epstopdf
	\item hyperref
	\item microtype
\end{itemize}

\end{enumerate}

A great (in my opinion) integrated {\LaTeX} environment for Windows is TeXnicCenter (\url{http://www.texniccenter.org/}). A great help for writing {\LaTeX} can be found in the Wikibook \url{http://en.wikibooks.org/wiki/LaTeX}. 

\vspace{+3em}
Enjoy the writing. 

% MS Thesis
\includepdf[pages=-,
						pagecommand={},
            addtotoc={1, section, 1, Example MS Thesis, sec:ms_thesis},
            ]{ms_thesis.pdf}
						%trim=0 20 0 50, clip

\includepdf[pages=-,
						pagecommand={},
            addtotoc={1, section, 1, Example PhD Dissertation, sec:phd_diss},
            ]{phd_dissertation.pdf}

\end{document}
%%
%% End of file `pdf-toc.tex'.
